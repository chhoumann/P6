Using weather data gathered from different cyber physical systems in the form of weather sensors, we have trained and compared several different techniques for time series forecasting.
Our main focus was to show that the transformer would outperform previous state-of-the-art statistical and machine learning techniques.
As can be seen in the \nameref{sec:ExpRes} section, the transformer indeed has the lowest RMSE error compared to the other models on every dataset. 
It is worth noting that the transformer used here deviates slightly from the original transformer model proposed by \citet{AttentionIsAllYouNeed}, as we have removed the decoder and implemented the time embedding module presented by \citet{time2vec}.
These are described in detail in section \ref{sec:system architecture}.

We have developed a web application to show these results in a simple graphical user interface. 
The user of the application is able to make temperature predictions by selecting a model and dataset from a dropdown menu. 

In the future, it would be interesting to study the reason for the RNN outperforming the LSTM model.
Based on the \nameref{sec:relatedwork} section, we expected the LSTM model to outperform the RNN model on these larger datasets.
One could run these on more and larger datasets to investigate whether this is still the case. 
In addition, it could also be beneficial to experiment further with hyperparameter tuning for all models.

In conclusion, the results from our experiments present new evidence suggesting that the transformer is better suited for time series forecasting than previous state-of-the-art techniques.