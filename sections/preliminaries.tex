\subsection{Time series problem definition}
Given a cyber-physical system with \textit{N} sensors, each sensor is attributed with a time series.
Each timestamp in each of these time series contains information about \textit{C} features.
A time series of \textit{N} sensors can then be represented as \(X \in \mathbb{R}^{N \times C \times T}\).
The time series from the \textit{i}-th sensor is captured by \(x^{(i)} \in \mathbb{R}^{C \times T} \).
In addition, \(x_{t} \in \mathbb{R}^{N \times C}\) captures the time series of \textit{N} sensors at timestamp \textit{t} with \textit{C} features.
Finally, the vector of attributes for the \textit{i}-th sensor at timestamp \textit{t} is represented with \(x_{t}^{(i)} \in \mathbb{R}^{C}\). \cite{cirsteaEnhanceNetPluginNeural2021}

% maybe TODO: Define correlated time series forecasting, if we end up using it.
% TODO: Formal problem definition 

\subsection{Problem overview}
Given a time series as previously mentioned, it is useful to be able to forecast the future behavior of the time series.
This can be done by using a model that is able to predict the future behavior of the time series.
That is, we predict some target feature value at a future time step, e.g., the next value of the temperature.


There exist many techniques to do such forecasting. Some of these were presented in section \ref{sec:relatedwork}.
Given these models, there must be a way to compare their performance.
Therefore, we will compare the performance of the previously mentioned methods to the more recent transformer model.
The task is to predict the mean temperature in a time series at a future time step, given data about previous time steps.