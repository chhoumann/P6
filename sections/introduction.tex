\label{sec:intro}
A cyber-physical system (CPS) is a computerized system where real world physical events or mechanisms are monitored or controlled through the application of computer algorithms.
In a CPS, physical and software components work together in different spatial and temporal scales.

An example of such a system is how large weather sensor networks work in correlation. These produce both local weather time series, but they also affect how data from other sensors are interpreted. This, in turn, produces multiple weather time series that are all correlated. Being able to accurately forecast the weather has a large impact on most other CPSs as well as most people's daily lives. Having an accurate forecasting model is necessary in order to be able to accurately identify trends and outliers as well as predicting the future behavior of the weather. Ultimately all of this is useful in both the day to day running of other CPSs, and also in predicting future possible climate changes.\cite{alurPrinciplesCyberphysicalSystems2015} 

Weather is a suitable problem to model as a time series since historical weather data is timestamped.
Forecasting can then be done using a forecasting model.
The major advantage to time series forecasting is that there is no need to build complex physical systems for measuring the weather. 
Previous data is simply used to predict future behavior.

To generate a reliable time series weather forecasting model, we need a model that is able to account for the seasonality aspect of the weather and the variable attributes that affect the future. 
This means that, in order to generate an accurate model, one must be able to take into account previous historical states when making predictions for the future. 

Today, the state of the art methods in time series forecasting include methods such as autoregressive integrated moving average (ARIMA), recurrent neural networks (RNNs), long short term memory (LSTM) and gated recurrent units
(GRUs).
However, each of these models have limitations that make them problematic to use in time series forecasting.
For example, RNNs have the vanishing gradient problem, and LSTMs and GRUs use many parameters, leading to a complex model.

More recently, in the field of natural language processing (NLP), the transformer model has shown great potential and does not suffer from similar issues as the aforementioned methods, and is even able to outperform them \cite{AttentionIsAllYouNeed}.
Because of this, we wish to compare the performance of the previously used methods to the more recent transformer model when performing univariate time series weather forecasting. Our goal is to determine whether this model is able to also outperform the previously used methods in time series forecasting.

The original transformer model was intended for use in NLP.
We use a modified version that can be used reliably for time series forecasting.
This architecture has been used to train a model on weather datasets from different cities around the world.
Its performance was then compared to the performance of other forecasting models on the same datasets. 
Using the results from this, we have developed a web application that allows a user to predict the weather at a given date using all the trained models.
This is done through a pipeline; data from various datasets, described in section \ref{sec:ExpRes}, first go through a preprocessing phase where they are filtered and prepared for use.
After this, they are used in the desired model, and finally the resulting prediction is displayed to the user on the website.  

This paper presents a web application for a weather forecasting system.
We also detail a modified transformer model inspired by \citet{schmitz_stock_2020} capable of using temporal data, which has been compared to conventional time series forecasting techniques.
