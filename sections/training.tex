\subsection{Training}\label{sec:training}
The training was done using individual Python notebooks contained in a project on Deepnote\cite{deepnote} in order to allow easy collaboration between the authors.

The data used are open source and was collected on Kaggle\cite{kaggle}. It was provided by individual contributors for free use.
As such, it is worth noting that the data therefore came from arbitrary cities in the world, and that we are unable to verify its legitimacy.
This should be of no concern, however, since the primary goal was to use meaningfully large and representative datasets for training the models.

Once the models have been trained, they are serialized using the Python \texttt{pickle} module.
This module converts the data and the Python objects into a byte stream which is stored in a file.
The process of serializing Python objects is known as pickling. 
This allows us to export the trained models from Deepnote, the first component in our pipeline, and onto the server used in the web application, the second component in our pipeline, as previously shown in figure \ref{fig:architecture diagram}.\cite{pickle_documentation}