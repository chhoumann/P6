\subsection{Training}\label{sec:training}
In order to train the models, we used Deepnote\cite{deepnote} to achieve similar hardware performance on the training of all the models used for this project. 
The training was done using individual Python notebooks contained in a project on Deepnote, in order to allow easy collaboration between the authors.

The data used are open source and was collected on Kaggle\cite{kaggle}. It was provided by individual contributors for free use.
The data came from arbitrary cities of the world. 
It should be noted that we cannot verify the legitimacy of the data, the main objective was simply to gather datasets that were representative and meaningfully large to train the models on. 

Once the models have been trained, they are serialized using the Python \texttt{pickle} module.
This module converts the data and the Python objects into a byte stream which is stored in a file.
The process of serializing Python objects is known as pickling. 
This allows us to export the trained models from Deepnote, the first component in our pipeline, and onto the server used in the web application, the second component in our pipeline, as previously shown in figure \ref{fig:architecture diagram}.\cite{pickle_documentation}