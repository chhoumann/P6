\subsection{Data loading and preprocessing}\label{sec:data loading and preprocessing}
In order to load the data, we use the Pandas library for Python.
Data are loaded from CSV files and then passed to the preprocessing stage.

Preprocessing is done by first stripping the Pandas DataFrame of all columns except the target column $y$. In our case, this is the column denoting the temperature.

Next, we add $n$ time lag features. Let $x_{t,y}$ denote value of the target attribute $y$ at time step $t$ for $N$ sensors. 
Time lag features are added by adding $L=\{1,\dots, n\}$ attributes where $L_{i=1}^n=x_{t-i, y}$ for each $x_{t}$.

As we used different Python frameworks for different models, the processing from here depended on the framework. Some models required the dataset to be converted to tensors, and others did not.
Besides this, we also split the data into test and training sets with a $0.33/0.77$ distribution for testing and training, respectively.
